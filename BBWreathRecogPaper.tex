\documentclass[
12pt, % The default document font size, options: 10pt, 11pt, 12pt
%oneside, % Two side (alternating margins) for binding by default, uncomment to switch to one side
english, % ngerman for German
singlespacing, % Single line spacing, alternatives: onehalfspacing or doublespacing
%draft, % Uncomment to enable draft mode (no pictures, no links, overfull hboxes indicated)
%nolistspacing, % If the document is onehalfspacing or doublespacing, uncomment this to set spacing in lists to single
%liststotoc, % Uncomment to add the list of figures/tables/etc to the table of contents
%toctotoc, % Uncomment to add the main table of contents to the table of contents
%parskip, % Uncomment to add space between paragraphs
%nohyperref, % Uncomment to not load the hyperref package
headsepline, % Uncomment to get a line under the header
%chapterinoneline, % Uncomment to place the chapter title next to the number on one line
%consistentlayout, % Uncomment to change the layout of the declaration, abstract and acknowledgements pages to match the default layout
]{scrartcl} % The class file specifying the document structure

\usepackage[utf8]{inputenc} % Required for inputting international characters
\usepackage[T1]{fontenc} % Output font encoding for international characters

\usepackage{amsmath}
\usepackage{amssymb}
\usepackage{amsthm}

\usepackage{amsfonts, amscd}
% \usepackage[usestackEOL]{stackengine}
% \newcommand\fop[3][9pt]{\mathop{\ensurestackMath{\stackengine{#1}%
%   {\displaystyle#2}{\scriptstyle#3}{U}{c}{F}{F}{L}}}\limits}
% \newcommand\finf[2][9pt]{\fop[#1]{\inf}{#2}}
% \newcommand\fsum[2][13pt]{\fop[#1]{\sum}{#2}}


%%%%%%%%%%%%%%%%%%%%%%%%%%%%%%%%%%%%%%%%%%%%%% my packages %%%%%%%%%%%%%%%%%%%%%%%%%%%%%%%%%%%%%%%%%%%%%%%%%%%%%%%%%%%%%%%%%%%%%%%%%%%%%%%%%%%%%%%%%%%%%%%%%%%%%%%
\usepackage{mathtools}

\usepackage{todonotes}

\usepackage{float}

\usepackage{comment}

\usepackage{enumitem}

\usepackage{algorithm}
\usepackage[noend]{algpseudocode}

\usepackage{xcolor}

\def\mathunderline#1#2{\color{#1}\underline{{\color{black}#2}}\color{black}}

\usepackage{tikz}

\usepackage{calculator}

\usepackage[normalem]{ulem}

\usepackage{hyperref}

\usepackage{ifthen}
\newboolean{Notizen} %Deklaration
\setboolean{Notizen}{true} %Zuweisung
\newcommand{\notiz}[1]{\ifthenelse{\boolean{Notizen}}{ \todo[inline]{#1}}{}}
\newboolean{Notizen2} %Deklaration
\setboolean{Notizen2}{true} %Zuweisung
\newcommand{\notizseite}[1]{\ifthenelse{\boolean{Notizen2}}{ \todo[noline]{#1}}{}}

\newcounter{counter}
\newtheorem{rem}[counter]{Remark}
\newtheorem{notation}[counter]{Notation}
\newtheorem{thm}[counter]{Theorem}
\newtheorem{defn}[counter]{Definition}
\newtheorem{lem}[counter]{Lemma}
\newtheorem{cor}[counter]{Corollary}
\newtheorem{ex}[counter]{Example}


\newcounter{nameOfYourChoice}

\newcommand{\fa}{for all }
\newcommand{\iaoi}{if and only if }
\newcommand{\io}{\overline{\imath}}

\newcommand{\divides}{\bigm|}
\newcommand{\ndivides}{%
  \mathrel{\mkern.5mu % small adjustment
    % superimpose \nmid to \big|
    \ooalign{\hidewidth$\big|$\hidewidth\cr$\nmid$\cr}%
  }%
}

\DeclareMathOperator{\SC}{SC}
\DeclareMathOperator{\Fix}{Fix}
\DeclareMathOperator{\Prop}{Prop}
\DeclareMathOperator{\Prob}{P}
\DeclareMathOperator{\R}{R}
\DeclareMathOperator{\Z}{Z}
\DeclareMathOperator{\Prime}{\mathbb{P}}
\DeclareMathOperator{\Ered}{E_{red}}
\DeclareMathOperator{\EB}{E_{B}}
\DeclareMathOperator{\E}{E}
\DeclareMathOperator{\Eneg2}{E_{\neg 2}}
\DeclareMathOperator{\N}{\mathbb{N}}
\DeclareMathOperator{\Aut}{Aut}
\DeclareMathOperator{\Inn}{Inn}
\DeclareMathOperator{\Det}{Det}
\DeclareMathOperator{\det2}{det}
\DeclareMathOperator{\Sym}{Sym}
\DeclareMathOperator{\supp}{supp}
\DeclareMathOperator{\fix}{fix}
\DeclareMathOperator{\terr}{terr}
\DeclareMathOperator{\lcm}{LCM}
\DeclareMathOperator{\Stab}{Stab}
\DeclareMathOperator{\ifff}{\text{ if and only if }}

\allowdisplaybreaks[1]
%%%%%%%%%%%%%%%%%%%%%%%%%%%%%%%%%%%%%%%%%%%%%%%%%%%%%%%%%%%%%%%%%%%%%%%% end of my packages %%%%%%%%%%%%%%%%%%%%%%%%%%%%%%%%%%%%%%%%%%%%%%%%%%%%%%%%%%%%%%%%%%%%%%%%%%%%%%%%%%%%%%%%%%%%%%%%%

\usepackage{mathpazo} % Use the Palatino font by default

\usepackage[backend=bibtex,natbib=true]{biblatex} % Use the bibtex backend with the authoryear citation style ( style=authoryear removed ) (which resembles APA)

\addbibresource{bibliography.bib} % The filename of the bibliography

\usepackage[autostyle=true]{csquotes} % Required to generate language-dependent quotes in the bibliography

%----------------------------------------------------------------------------------------
%	MARGIN SETTINGS
%----------------------------------------------------------------------------------------


%%%% War für alte Documentclass MastersDoctoralThesis
% \geometry{
% 	paper=a4paper, % Change to letterpaper for US letter
% 	inner=2.5cm, % Inner margin
% 	outer=3.8cm, % Outer margin
% 	bindingoffset=.5cm, % Binding offset
% 	top=1.5cm, % Top margin
% 	bottom=1.5cm, % Bottom margin
% 	%showframe, % Uncomment to show how the type block is set on the page
% }


%----------------------------------------------------------------------------------------
%	TITLEPAGE SETTINGS
%----------------------------------------------------------------------------------------

\title{A Black Box Recognition Algorithm for Wreath Products}

\begin{document}
\maketitle

\begin{abstract}
%\addchaptertocentry{\abstractname} % Add the abstract to the table of contents
\vspace{2cm}
We present a one-sided Monte Carlo algorithm which, for a given black box group $G$,
decides whether or not $G$ is isomorphic to a wreath product $A_n\wr H$, where $H$ is a transitive subgroup of the symmetric group on $\Gamma$ and $n\geq 9$.
If the algorithm proves $G$ to be isomorphic to $A_n\wr H$, it will return an explicit isomorphism.

As the methods given in this thesis are mostly probabilistic, we rely on estimates of certain proportions
of elements in wreath products $K\wr_\Gamma H$.
%We show that we can estimate these proportions against proportions in $K$.
In particular we show that the proportion of elements $g\in K\wr_\Gamma H$, which power to involutions in the base group, is at least as high as the proportion of
elements of order divisible by 2 in $K$.
As this and most other observations in this thesis do not require $K$ to be a specific permutation group, this leaves room for generalizing the methods presented to other types of groups.
\vspace{2cm}
\vspace{2cm}
% 
% 
% We present a recognition algorithm to decide whether
% a given black-box group is isomorphic to an alternating or a
% symmetric group without prior knowledge of the degree. This
% eliminates the major gap in known algorithms, as they require the
% degree as additional input.
% Our methods are probabilistic and rely on results about proportions
% of elements with certain properties in alternating and symmetric
% groups. These results are of independent interest; for instance, we
% establish a lower bound for the proportion of involutions with
% small support.
% 
% 
% Abstract. We present a Las Vegas algorithm which, for a given black-box
% group known to be isomorphic to a symmetric or alternating group, produces
% an explicit isomorphism with the standard permutation representation of the
% group. This algorithm has applications in computations with matrix groups
% and permutation groups.
% In this paper, we handle the case when the degree n of the standard per-
% mutation representation is part of the input. In a sequel, we shall treat the
% case when the value of n is not known in advance.
% As an important ingredient in the theoretical basis for the algorithm, we
% prove the following result about the orders of elements of S n : the conditional
% probability that a random element σ ∈ S n is an n-cycle, given that σ n = 1, is
% at least 1/10.
\end{abstract}

\include{Sections/section1}
\section{Background}


\subsection{The Problem}


\subsection{Black Box Groups}

The term \emph{black box group} was introduced by Babai and Szemerédi in \cite{babai1984complexity} and it refers to a way of representing a group within a computer where the black box performs the group operations. 
This is the most general representation of a group.
The algorithm presented in this paper constructively recognizes a wreath product $A_n\wr_\Gamma H$ given as a black box group where $H \leq S_\Gamma$ transitive.
Such an algorithm then works on $A_n\wr H$ given in any representation e.g. as matrices or permutations.
\begin{defn}
A \emph{black box group} is a group $G$ whose elements are encoded as bit-strings of uniform length $N$.
The black box can compute strings for $gh$ and $g^{-1}$ or decide whether $g=h$ for any $g,h \in G$.
\end{defn}
Note that each bit-string represents at most one element of $G$ but an element of $G$ can have multiple representations as bit-strings.
Thus $|G| \leq 2^N$. 

\subsection{Notation}

\section{Outline of the Algorithm}
\notiz{noch 1:1 aus der MA}
We briefly outline the steps of the algorithm described in this thesis.
The goal is to constructively recognize a black box group $G$, given it is isomorphic to a wreath product $A_n \wr_\Gamma H$ with $H\leq S_\Gamma$ transitive.
 
 
 \begin{itemize}[leftmargin=3.0cm,labelsep=0.5cm]
\item[\textbf{INPUT:}] a black-box group $G = \langle X \rangle$ encoded as bit strings of uniform
length $N$ and a real number $\epsilon$ with $0 < \epsilon < 1$.
\item[\textbf{OUTPUT:}] an isomorphism $\varphi \colon G\to A_n\wr_\Gamma H$ if $G$ is isomorphic to \mbox{$A_n\wr_\Gamma H$}
with $H\leq S_\Gamma$ transitive, else failure.
\end{itemize}
In the following description we assume that $G$ is isomorphic to a wreath product $A_n \wr_\Gamma H$ with $H\leq S_\Gamma$ transitive.
If this is not the case, then either one of the steps will fail, or the correctness check in Step~\ref{stepcorr} will report \texttt{false}.
Further note that we repeat this sequence of steps multiple times, until we either find an isomorphism or with sufficiently high probability $G$ is not
isomorphic to a wreath product $A_n \wr_\Gamma H$ with $H\leq S_\Gamma$ transitive.
\begin{enumerate}
\item \label{stepone}
By taking random elements of $G$, find an element $g$ such that $g^{\frac{|g|}{2}}$ is an element of the base group.
As we are in a black box situation, we can not decide whether or not an element is in the base group.
Thus we continue the computation until an error occurs.
In that case we try with the next element, unless we have tried sufficiently many times such that we can be sure with probability at least $1-\epsilon$ that $G$ is not
isomorphic to a wreath product $A_n \wr_\Gamma H$ with $H\leq S_\Gamma$ transitive.
In Lemma~\ref{lemmapropbasered}
%Let $n\geq 6$, $G=K\wr_\Gamma H$ and $H\leq S_m$. % and $P=\{g\in G \mid 2\text{ divides } |g| \text{ and } g^{\frac{|g|}{2}} \in B\}$.
we show that for $n\geq 6$ the proportion of elements $g$ in $A_n\wr_\Gamma H$ such that $g^{\frac{|g|}{2}}$ is an element of the base group is at least $\frac{3}{8}_.$
\item With Lemma~\ref{directproductnts} we show that the only normal subgroups $A$ of the base group $B\cong A_n^\Gamma$ are isomorphic to $A_n^s$ with territory size $s$ for $0\leq s \leq |\Gamma|$.
This also yields that, given $H$ is transitive on $\Gamma$, the normal closure of an element of the base group under $A_n\wr_\Gamma H$ is equal to the base group.
Having the base group computed, we try to reduce the size of the territory by taking random elements $g$ of the current subgroup $A\cong A_n^s$
and computing the normal closure of $g^{\frac{|g|}{2}}$ under $A$.
Lemma~\ref{Rabschätzung} shows that for $n\geq 6$ and $s\geq 2$ the proportion of elements $g$ in $A_n^s$
such that $g^{\frac{|g|}{2}}$ is in $A_n^t$ for $1\leq t <s $ is at least $\frac{1}{10}\frac{1}{\sqrt n}$.
Hence we repeat this process until with sufficiently high probability we obtain a subgroup isomorphic $S$ to $A_n$ with singleton territory.
\item In Section~\ref{toprec} we show that the top group $(1_{A_n^\Gamma}, H)$ of $A_n\wr_\Gamma H$ acts on the singleton territories precisely as $H$ acts on $\Gamma$.
Thus, given a subgroup isomorphic to $A_n$ with singleton territory, we can determine a $G$-set $\Gamma'=\{S^{t_1},\dots,S^{t_m}\}$ for $t_1,\dots,t_m\in G$.
By the transitivity of $H$ this set is in bijection to $\Gamma$.
This way we can identify the top element of any $g\in G$.
\item \label{recstep}
We use the recognition algorithm by Jambor et al. $2013$ \cite{jambor2013fast} to check if $S$ is isomorphic to an alternating group of degree $n\geq 9$ and determine $n$.
If this is not the case we start from Step~\ref{stepone} again.
As shown in Section~\ref{recbase}, we can use the results of the algorithm by Jambor et al. %to fully identify any element of the wreath product, given we have identified the top element before.
to construct a map $\varphi$ from $G$ to $A_n \wr_\Gamma S_\Gamma$.
If indeed $G$ is isomorphic to $A_n \wr_\Gamma H$ with $H\leq S_\Gamma$ transitive and $S\cong A_n$ is a single component subgroup, then $\varphi$ is an isomorphism which we proceed to check in the next step.
\item \label{stepcorr}
\begin{sloppypar}
As a last step we perform a correctness check and for now assume that we are uncertain, whether or not $G$ is isomorphic to a wreath product $A_n \wr_\Gamma H$ with $H\leq S_\Gamma$ transitive.
For this check we first compute the centralizer $C$ in $G$ of a "diagonal subgroup" \mbox{$D^*=\{s^{t_1}\dots s^{t_m}\mid s\in S\}$}, as described in Section~\ref{comptop}.
We check whether for all $i\in \{1,\dots, m\}$ and $c\in C$ there is a $j\in \{1,\dots, m\}$ such that \mbox{$(s^{t_i})^c=s^{t_j}$} \fa $s \in S$.
If that is the case, then $C$ acts on $\Gamma'$ by Lemma~\ref{lemabove} and by Corollary~\ref{corabove} we have $(\langle Y\rangle)\varphi = A_n\wr_{\Gamma'} C^{\Gamma'}$ for $\langle S^{t_1}, \dots , S^{t_m}, C\rangle =\langle Y\rangle$.
Further we compute the centralizer of $\langle S^{t_1}, \dots , S^{t_m} \rangle$ in $C$ and check if it is trivial, which is precisely the case when $C$ acts faithfully on $\Gamma'$ by
Lemma~\ref{lemmatreu}.
In this case $\varphi$ is an isomorphism from $\langle Y \rangle$ to $A_n\wr_{\Gamma'} C^{\Gamma'}$.
% Note that, given $G$ is isomorphic to $A_n \wr_\Gamma H$, then $C$ is the top group of 
% If $G$ is indeed isomorphic to $A_n \wr_\Gamma H$ with $H\leq S_\Gamma$ transitive and our probabilistic computations up to this point are correct, then $C$ is isomorphic to
% the top group $(1_{A_n^\Gamma}, H)$ and $\langle S^{t_1},\dots, S^{t_m}\rangle$ is isomorphic to the base group $(K^\Gamma,1_H)$.
% Using $C$ and $S^{t_1},\dots, S^{t_m}$, we try to construct generators $X$ such that the map $\varphi$ computed in Step~\ref{recstep} induces an isomorphism 
% from $\langle X \rangle$ to $A_n \wr_\Gamma H$ for some $H\leq S_\Gamma$.
Hence we found an isomorphism from $G$ to $A_n\wr_{\Gamma'} C^{\Gamma'}$, precisely if $\langle Y \rangle=G$.
Thus, for each generator $g$ of $G$ we compute the image under $\varphi$ and compute the unique preimage in $\langle Y \rangle$ of $(g)\varphi$ under $\varphi$,
which we can do efficiently, as shown in Sections~\ref{sectioncomprebase} and \ref{sectiontoppreimage}.
If all generators of $G$ equal their respective preimage, then $Y$ generates $G$, which means $\varphi$ passes the correctness test and we return \texttt{true} together with the isomorphism $\varphi$.
Note that, if $G$ is isomorphic to $A_n \wr_\Gamma H$ with $H\leq S_\Gamma$ transitive and $S\cong A_n$ is a single component subgroup, then $\varphi$ is an isomorphism and $S$ and $\varphi$ pass the correctness check.
\end{sloppypar}

% Either way, if the centralizer computation suceeded, then $\langle S^{t_1},\dots, S^{t_m},C\rangle$ forms a wreath product isomorphic $S\wr_{\Gamma'} C$ by Lemma~\ref{corabove} and the
% 
\end{enumerate}

\include{Sections/section4}
\section{Proportions In Wreath Products}

%%%%%%%%%%%%%%%%%%%%%%%%%%%%%%%%%%%%%%%%%%%%%%%%%%%%%%%%%%%%%%%%%%%%%%%%%%%%%%%%%%%%%%%%%%%%%%%%%%%%%%%%
\notiz{ab hier für Lemma \ref{lemmapropbasered}}
%%%%%%%%%%%%%%%%%%%%%%%%%%%%%%%%%%%%%%%%%%%%%%%%%%%%%%%%%%%%%%%%%%%%%%%%%%%%%%%%%%%%%%%%%%%%%%%%%%%%%
\begin{defn}
Let $K$ be a group and $q\in \N$.
We define
\begin{align*}
\E_{\neg q}(K)&=\{k \in K\mid q \text{ does not divide } |k|\}\\
\text{ and } \E_{q}(K)&=\{k \in K\mid q \text{ divides } |k|\}.
\end{align*}
\end{defn}

\notiz{das Lemma hier muss ohne die umständliche Notation auskommen, verweise auf ORE}
\begin{lem}\label{concrete}
Let $G=K\wr_\Gamma H$, $H\leq S_m$ and $\tilde{G}=K\wr_\Gamma S_\Gamma$. % and $P=\{g\in G \mid 2\text{ divides } |g| \text{ and } g^{\frac{|g|}{2}} \in B\}$.
Then \[\frac{|\EB(G)|}{|G|}\geq\frac{|\E_2(K)|}{|K|}_.\]
\end{lem}
\begin{proof}
For each $\pi \in H$ we construct a subset $\E_{B}(\pi)$ of $\EB(G)$ containing only base elements with top element $\pi$, which has cardinality $|\E_2(K)||K|^{m-1}$.
Then the disjoint union of said subsets has cardinality $|H||\E_2(K)||K|^{m-1}$, which yields the result.
First let $1\neq \pi\in H$ with disjoint cycle representation $\pi_1\dots\pi_s$ ordered such that $|\pi_1|_2\geq\dots\geq |\pi_s|_2$. 
%Further let $E_2(K^\Gamma)=\{f \in K^\Gamma \mid 2\text{ divides } |f|\}$ and note that for any $\gamma \in \supp(\pi_1)$ we have
%$\{c\in \SC_{\tilde{G}}(\pi_1) \mid \delta_{\pi_1}(c) \in E_2(K^\Gamma)\}=\{c\in \SC_{\tilde{G}}(\pi_1) \mid \delta_{\pi_1,\gamma}(c) \in E_2(K)\}.$
%In $G$ there are $K^{|\Ga|}$ elements with top element $\pi$.
%For $\pi \neq 1_H$ we have
Let
\begin{align*}
&\E_{B}(\pi)\\
&=\!\!\{c\in \SC_{\tilde{G}}(\pi_1) \! \mid \! (c)\Det \in \E_2(K^\Gamma)\} \!\times \! \SC_{\tilde{G}}(\pi_2)\! \times \! \dots \! \times \! \SC_{\tilde{G}}(\pi_s) \! \times \! 
B_{\neg \supp(\pi)}(G).\end{align*}
First we show that $\E_{B}(\pi)$ is a subset of $\EB(G)$.
For $g\in \E_{B}(\pi)$ we can write $g$ as product of disjoint strongly caged elements such that $c_i=(f_i,\pi_i)$ where $\pi_i=1_H$ for $s+1\leq i\leq \ell$. 
By Lemma~\ref{orderlemma} and the way we ordered the cycles we have
\begin{align*} &2\lcm(|\pi_1|,\dots,|\pi_s|)=\lcm(2|\pi_1|,\dots,|\pi_s|)\\\text{ divides } &\lcm(|(c_1)\Det||\pi_1|,\dots,|(c_\ell)\Det||\pi_\ell|)=|g|. \end{align*} 
Hence \mbox{$c_i^{\frac{|g|}{2}}\in B$} by Lemma~\ref{divides} and thus $g\in \E_B$.

\begin{sloppypar}
To show $\E_{B}(\pi)$ has the desired cardinality we fix a $\gamma\in \supp(\pi_1)$.
By Corollary~\ref{detordercor} we have $|(c_1)\Det|=|(c_1)\Det_\gamma|$ and thus
\mbox{$\{c\in \SC_{\tilde{G}}(\pi_1) \mid (c)\Det \in \E_2(K^\Gamma)\}$}$=$\mbox{$\{c\in \SC_{\tilde{G}}(\pi_1) \mid (c)\Det_\gamma \in E_2(K)\}$}.
Hence by Lemma~\ref{probshiftcor} we have $|\E_{B}(\pi)|=|\E_2(K)||K|^{m-1}$.

Otherwise, for $\pi=1_H$ let
\[\E_{B}(1_H)=\{c\in \SC_{\tilde{G}}(1_H) \mid (c)\Det_1 \in \E_2(K)\}\times B_{\neg \{1\}}(G). \] 
We have $|\{c\in \SC_{\tilde{G}}(1_H) \mid (c)\Det_1 \in \E_2(K)\}|=|\E_2(K)|$
and thus \mbox{$|\E_{B}(1_H)|=|\E_2(K)||K|^{m-1}$}.
Further $\E_{B}(1_H) \subseteq \EB(G)$ as $\E_{B}(1_H) \subseteq B$ and $2$ divides any element of $\E_{B}(1_H)$.
Finally, for $\pi,\sigma\in H$ with $\pi \neq \sigma$ we have \mbox{$\E_{B}(\pi)\cap\E_{B}(\sigma)=\emptyset$} as $\E_{B}(\pi)$ and $\E_{B}(\sigma)$ contain exclusively elements with top element $\pi$ and $\sigma$ respectively.

Together this yields \[|\EB(G)|\geq\sum_{\pi\in H}|\E_{B_\pi}|=\sum_{\pi\in H}|\E_2(K)||K|^{m-1}=|H||\E_2(K)||K|^{m-1}\]
%For $\pi=1$ we have $\{c\in \SC_{\tilde{G}}(\pi_1) \mid \delta_{\pi_1,1}(c) \in E_2\}\subseteq P$ and $|\{c\in \SC_{\tilde{G}}(\pi_1) \mid \delta_{\pi_1,1} \in E_2\}|=|E_2|$. 
% Together with Lemma~\ref{probshiftcor} this yields
% \begin{align*}
% |P|\geq&
% \sum_{\pi\in H\backslash 1_H} |\{c\in \SC_{\tilde{G}}(\pi_1) \mid \delta_{\pi_1}(c) \in E_2(K^\Gamma)\} \times \SC_{\tilde{G}}(\pi_2) \times \dots \times \SC_{\tilde{G}}(\pi_s) \times \R_{\supp(\pi)}(G)|\\
% &~~~~+~~|\{c\in \SC_{\tilde{G}}(1_H) \mid \delta_{1_H,1} \in E_2(K)\}\times \R_{\{1\}}(G)|\\
% =&\sum_{\pi\in H}|E_2(K)||K|^{m-1}=|H||E_2(K)||K|^{m-1}
% \end{align*}
and thus
\[\frac{|\EB(G)|}{|G|}=\frac{|\EB(G)|}{|G|}\geq\frac{|H||\E_2(K)||K|^{m-1}}{|H||K|^{m}}=\frac{|\E_2(K)|}{|K|}=\frac{|\E_2(K)|}{|K|}_.\]
%For $\pi=1$ we have $pi_1=1$ and therefore $|\{(f,\pi_1)\in \SC_{\tilde{G}}(\pi_1) \mid \tilde{f}_\gamma \in E_2\}|=
\end{sloppypar}
\end{proof}



\begin{thm}\label{engesnabschätzung}
Let $q\geq 2$ and $n\geq 1$. Then
%and $c(q)=\Gamma(1-\frac{1}{q})^{-1}$, where $\Gamma$ is the well-known $\Gamma$ function as for example defined in Chapter 12 of \emph{"A course of modern analysis"} 
%{\normalfont\cite{whittaker1996course}} by E. T. Whittaker and G. N. Watson.

%\begin{enumerate}[label=(\alph*)]
%\item
$\frac{|\E_{\neg q}(S_n)|}{|S_n|}=\prod_{\ell=1}^{\lfloor\frac{n}{q}\rfloor}(1-\frac{1}{\ell q})_.$
%\item if $n\geq q$, then $c(q)(\frac{q}{n})^{\frac{1}{q}}(1-n^{-1})\leq \frac{|E_{\neg q}(S_n)|}{|S_n|} \leq c(q)(\frac{q}{n})^{\frac{1}{q}}(1+2n^{-1})_.$
%\end{enumerate}
\end{thm}
\begin{proof}
See \cite[Theorem 2.3 (a),%(b),
p.4]{beals2002permutations}.
\end{proof}


The proportion of elements not divisible by 2 within $A_n$ is precisely double the corresponding proportion in $S_n$ as the following Corollary shows.
\begin{cor}\label{doubleprop}
Let $n = 2m + r \geq 2$, where $0 \leq r \leq 1$. Then
\[\frac{|\Eneg2(A_n)|}{|A_n|}=2\frac{|\Eneg2(S_n)|}{|S_n|}_.\]
\end{cor}
\begin{proof}
By \cite[Theorem 3.3 (a), p.8]{beals2002permutations} we have
\begin{align*}
%a_{\neg q}(n)=s_{\neg q}(n)+(1-1/6)
\frac{|\E_{\neg 2}(A_n)|}{|A_n|}
&=\frac{|\E_{\neg 2}(S_n)|}{|S_n|}+(-1)^{2\lfloor\frac{n}{2}\rfloor} \prod_{\ell=1}^{\lfloor\frac{n}{2}\rfloor}(1-\frac{1}{2 \ell})\\
&=\frac{|\E_{\neg 2}(S_n)|}{|S_n|}+\prod_{\ell=1}^{\lfloor\frac{n}{2}\rfloor}(1-\frac{1}{2 \ell})=2\frac{|\Eneg2(S_n)|}{|S_n|}_.
\end{align*}
\end{proof}




\begin{lem}\label{lemmapropbasered}
Let $n\geq 6$, $G=A_n\wr_\Gamma H$ and $H\leq S_m$. % and $P=\{g\in G \mid 2\text{ divides } |g| \text{ and } g^{\frac{|g|}{2}} \in B\}$.
Then $\frac{|\EB(G)|}{|G|}\geq \frac{3}{8}_.$
\end{lem}
\begin{proof}
As $\E_2(A_n)=A_n\backslash \E_{\neg 2}(A_n)$ we have with Lemmata \ref{concrete}, \ref{engesnabschätzung} and \ref{doubleprop} that
\begin{align*}
\frac{|\EB(G)|}{|G|}&\geq\frac{|\E_2(A_n)|}{|A_n|}=1-\frac{|\Eneg2(A_n)|}{|A_n|}= 1-2\frac{|\Eneg2(S_n)|}{|S_n|}\\
&=1-2\prod_{\ell=1}^{\lfloor\frac{n}{2}\rfloor}(1-\frac{1}{2 \ell})
\geq 1-2\prod_{\ell=1}^{\lfloor\frac{6}{2}\rfloor}(1-\frac{1}{2 \ell})\\
&=1-2(1-\frac{1}{2})(1-\frac{1}{4})(1-\frac{1}{6})=1-\frac{10}{16}=\frac{6}{16}=\frac{3}{8}_.
\end{align*}
\end{proof}


%%%%%%%%%%%%%%%%%%%%%%%%%%%%%%%%%%%%%%%%%%%%%%%%%%%%%%%%%%%%%%%%%%%%%%%%%%%%%%%%%%%%%%%%%%%%%%%%%%%%%%%%%%%%%%%%%%%%%%%%%%%%%%%%%%%%%

%%%%%%%%%%%%%%%%%%%%%%%%%%%%%%%%%%%%%%%%%%%%%%%%%%%%%%%%%%%%%%%%%%%%%%%%%%%%%%%%%%%%%%%%%%%%%%%%%%%%%%%%%%%%%%%%%%%%%%%%%%%%%%%%%%%%


\printbibliography[heading=bibintoc]

%----------------------------------------------------------------------------------------

\end{document}  
