\section{Background}


\subsection{The Problem}


\subsection{Black Box Groups}

The term \emph{black box group} was introduced by Babai and Szemerédi in \cite{babai1984complexity} and it refers to a way of representing a group within a computer where the black box performs the group operations. 
This is the most general representation of a group.
The algorithm presented in this paper constructively recognizes a wreath product $A_n\wr_\Gamma H$ given as a black box group where $H \leq S_\Gamma$ transitive.
Such an algorithm then works on $A_n\wr H$ given in any representation e.g. as matrices or permutations.
\begin{defn}
A \emph{black box group} is a group $G$ whose elements are encoded as bit-strings of uniform length $N$.
The black box can compute strings for $gh$ and $g^{-1}$ or decide whether $g=h$ for any $g,h \in G$.
\end{defn}
Note that each bit-string represents at most one element of $G$ but an element of $G$ can have multiple representations as bit-strings.
Thus $|G| \leq 2^N$. 

\subsection{Notation}
