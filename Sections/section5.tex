\section{Proportions In Wreath Products}

%%%%%%%%%%%%%%%%%%%%%%%%%%%%%%%%%%%%%%%%%%%%%%%%%%%%%%%%%%%%%%%%%%%%%%%%%%%%%%%%%%%%%%%%%%%%%%%%%%%%%%%%
\notiz{ab hier für Lemma \ref{lemmapropbasered}}
%%%%%%%%%%%%%%%%%%%%%%%%%%%%%%%%%%%%%%%%%%%%%%%%%%%%%%%%%%%%%%%%%%%%%%%%%%%%%%%%%%%%%%%%%%%%%%%%%%%%%
\begin{defn}
Let $K$ be a group and $q\in \N$.
We define
\begin{align*}
\E_{\neg q}(K)&=\{k \in K\mid q \text{ does not divide } |k|\}\\
\text{ and } \E_{q}(K)&=\{k \in K\mid q \text{ divides } |k|\}.
\end{align*}
\end{defn}

\notiz{das Lemma hier muss ohne die umständliche Notation auskommen, verweise auf ORE}
\begin{lem}\label{concrete}
Let $G=K\wr_\Gamma H$, $H\leq S_m$ and $\tilde{G}=K\wr_\Gamma S_\Gamma$. % and $P=\{g\in G \mid 2\text{ divides } |g| \text{ and } g^{\frac{|g|}{2}} \in B\}$.
Then \[\frac{|\EB(G)|}{|G|}\geq\frac{|\E_2(K)|}{|K|}_.\]
\end{lem}
\begin{proof}
For each $\pi \in H$ we construct a subset $\E_{B}(\pi)$ of $\EB(G)$ containing only base elements with top element $\pi$, which has cardinality $|\E_2(K)||K|^{m-1}$.
Then the disjoint union of said subsets has cardinality $|H||\E_2(K)||K|^{m-1}$, which yields the result.
First let $1\neq \pi\in H$ with disjoint cycle representation $\pi_1\dots\pi_s$ ordered such that $|\pi_1|_2\geq\dots\geq |\pi_s|_2$. 
%Further let $E_2(K^\Gamma)=\{f \in K^\Gamma \mid 2\text{ divides } |f|\}$ and note that for any $\gamma \in \supp(\pi_1)$ we have
%$\{c\in \SC_{\tilde{G}}(\pi_1) \mid \delta_{\pi_1}(c) \in E_2(K^\Gamma)\}=\{c\in \SC_{\tilde{G}}(\pi_1) \mid \delta_{\pi_1,\gamma}(c) \in E_2(K)\}.$
%In $G$ there are $K^{|\Ga|}$ elements with top element $\pi$.
%For $\pi \neq 1_H$ we have
Let
\begin{align*}
&\E_{B}(\pi)\\
&=\!\!\{c\in \SC_{\tilde{G}}(\pi_1) \! \mid \! (c)\Det \in \E_2(K^\Gamma)\} \!\times \! \SC_{\tilde{G}}(\pi_2)\! \times \! \dots \! \times \! \SC_{\tilde{G}}(\pi_s) \! \times \! 
B_{\neg \supp(\pi)}(G).\end{align*}
First we show that $\E_{B}(\pi)$ is a subset of $\EB(G)$.
For $g\in \E_{B}(\pi)$ we can write $g$ as product of disjoint strongly caged elements such that $c_i=(f_i,\pi_i)$ where $\pi_i=1_H$ for $s+1\leq i\leq \ell$. 
By Lemma~\ref{orderlemma} and the way we ordered the cycles we have
\begin{align*} &2\lcm(|\pi_1|,\dots,|\pi_s|)=\lcm(2|\pi_1|,\dots,|\pi_s|)\\\text{ divides } &\lcm(|(c_1)\Det||\pi_1|,\dots,|(c_\ell)\Det||\pi_\ell|)=|g|. \end{align*} 
Hence \mbox{$c_i^{\frac{|g|}{2}}\in B$} by Lemma~\ref{divides} and thus $g\in \E_B$.

\begin{sloppypar}
To show $\E_{B}(\pi)$ has the desired cardinality we fix a $\gamma\in \supp(\pi_1)$.
By Corollary~\ref{detordercor} we have $|(c_1)\Det|=|(c_1)\Det_\gamma|$ and thus
\mbox{$\{c\in \SC_{\tilde{G}}(\pi_1) \mid (c)\Det \in \E_2(K^\Gamma)\}$}$=$\mbox{$\{c\in \SC_{\tilde{G}}(\pi_1) \mid (c)\Det_\gamma \in E_2(K)\}$}.
Hence by Lemma~\ref{probshiftcor} we have $|\E_{B}(\pi)|=|\E_2(K)||K|^{m-1}$.

Otherwise, for $\pi=1_H$ let
\[\E_{B}(1_H)=\{c\in \SC_{\tilde{G}}(1_H) \mid (c)\Det_1 \in \E_2(K)\}\times B_{\neg \{1\}}(G). \] 
We have $|\{c\in \SC_{\tilde{G}}(1_H) \mid (c)\Det_1 \in \E_2(K)\}|=|\E_2(K)|$
and thus \mbox{$|\E_{B}(1_H)|=|\E_2(K)||K|^{m-1}$}.
Further $\E_{B}(1_H) \subseteq \EB(G)$ as $\E_{B}(1_H) \subseteq B$ and $2$ divides any element of $\E_{B}(1_H)$.
Finally, for $\pi,\sigma\in H$ with $\pi \neq \sigma$ we have \mbox{$\E_{B}(\pi)\cap\E_{B}(\sigma)=\emptyset$} as $\E_{B}(\pi)$ and $\E_{B}(\sigma)$ contain exclusively elements with top element $\pi$ and $\sigma$ respectively.

Together this yields \[|\EB(G)|\geq\sum_{\pi\in H}|\E_{B_\pi}|=\sum_{\pi\in H}|\E_2(K)||K|^{m-1}=|H||\E_2(K)||K|^{m-1}\]
%For $\pi=1$ we have $\{c\in \SC_{\tilde{G}}(\pi_1) \mid \delta_{\pi_1,1}(c) \in E_2\}\subseteq P$ and $|\{c\in \SC_{\tilde{G}}(\pi_1) \mid \delta_{\pi_1,1} \in E_2\}|=|E_2|$. 
% Together with Lemma~\ref{probshiftcor} this yields
% \begin{align*}
% |P|\geq&
% \sum_{\pi\in H\backslash 1_H} |\{c\in \SC_{\tilde{G}}(\pi_1) \mid \delta_{\pi_1}(c) \in E_2(K^\Gamma)\} \times \SC_{\tilde{G}}(\pi_2) \times \dots \times \SC_{\tilde{G}}(\pi_s) \times \R_{\supp(\pi)}(G)|\\
% &~~~~+~~|\{c\in \SC_{\tilde{G}}(1_H) \mid \delta_{1_H,1} \in E_2(K)\}\times \R_{\{1\}}(G)|\\
% =&\sum_{\pi\in H}|E_2(K)||K|^{m-1}=|H||E_2(K)||K|^{m-1}
% \end{align*}
and thus
\[\frac{|\EB(G)|}{|G|}=\frac{|\EB(G)|}{|G|}\geq\frac{|H||\E_2(K)||K|^{m-1}}{|H||K|^{m}}=\frac{|\E_2(K)|}{|K|}=\frac{|\E_2(K)|}{|K|}_.\]
%For $\pi=1$ we have $pi_1=1$ and therefore $|\{(f,\pi_1)\in \SC_{\tilde{G}}(\pi_1) \mid \tilde{f}_\gamma \in E_2\}|=
\end{sloppypar}
\end{proof}



\begin{thm}\label{engesnabschätzung}
Let $q\geq 2$ and $n\geq 1$. Then
%and $c(q)=\Gamma(1-\frac{1}{q})^{-1}$, where $\Gamma$ is the well-known $\Gamma$ function as for example defined in Chapter 12 of \emph{"A course of modern analysis"} 
%{\normalfont\cite{whittaker1996course}} by E. T. Whittaker and G. N. Watson.

%\begin{enumerate}[label=(\alph*)]
%\item
$\frac{|\E_{\neg q}(S_n)|}{|S_n|}=\prod_{\ell=1}^{\lfloor\frac{n}{q}\rfloor}(1-\frac{1}{\ell q})_.$
%\item if $n\geq q$, then $c(q)(\frac{q}{n})^{\frac{1}{q}}(1-n^{-1})\leq \frac{|E_{\neg q}(S_n)|}{|S_n|} \leq c(q)(\frac{q}{n})^{\frac{1}{q}}(1+2n^{-1})_.$
%\end{enumerate}
\end{thm}
\begin{proof}
See \cite[Theorem 2.3 (a),%(b),
p.4]{beals2002permutations}.
\end{proof}


The proportion of elements not divisible by 2 within $A_n$ is precisely double the corresponding proportion in $S_n$ as the following Corollary shows.
\begin{cor}\label{doubleprop}
Let $n = 2m + r \geq 2$, where $0 \leq r \leq 1$. Then
\[\frac{|\Eneg2(A_n)|}{|A_n|}=2\frac{|\Eneg2(S_n)|}{|S_n|}_.\]
\end{cor}
\begin{proof}
By \cite[Theorem 3.3 (a), p.8]{beals2002permutations} we have
\begin{align*}
%a_{\neg q}(n)=s_{\neg q}(n)+(1-1/6)
\frac{|\E_{\neg 2}(A_n)|}{|A_n|}
&=\frac{|\E_{\neg 2}(S_n)|}{|S_n|}+(-1)^{2\lfloor\frac{n}{2}\rfloor} \prod_{\ell=1}^{\lfloor\frac{n}{2}\rfloor}(1-\frac{1}{2 \ell})\\
&=\frac{|\E_{\neg 2}(S_n)|}{|S_n|}+\prod_{\ell=1}^{\lfloor\frac{n}{2}\rfloor}(1-\frac{1}{2 \ell})=2\frac{|\Eneg2(S_n)|}{|S_n|}_.
\end{align*}
\end{proof}




\begin{lem}\label{lemmapropbasered}
Let $n\geq 6$, $G=A_n\wr_\Gamma H$ and $H\leq S_m$. % and $P=\{g\in G \mid 2\text{ divides } |g| \text{ and } g^{\frac{|g|}{2}} \in B\}$.
Then $\frac{|\EB(G)|}{|G|}\geq \frac{3}{8}_.$
\end{lem}
\begin{proof}
As $\E_2(A_n)=A_n\backslash \E_{\neg 2}(A_n)$ we have with Lemmata \ref{concrete}, \ref{engesnabschätzung} and \ref{doubleprop} that
\begin{align*}
\frac{|\EB(G)|}{|G|}&\geq\frac{|\E_2(A_n)|}{|A_n|}=1-\frac{|\Eneg2(A_n)|}{|A_n|}= 1-2\frac{|\Eneg2(S_n)|}{|S_n|}\\
&=1-2\prod_{\ell=1}^{\lfloor\frac{n}{2}\rfloor}(1-\frac{1}{2 \ell})
\geq 1-2\prod_{\ell=1}^{\lfloor\frac{6}{2}\rfloor}(1-\frac{1}{2 \ell})\\
&=1-2(1-\frac{1}{2})(1-\frac{1}{4})(1-\frac{1}{6})=1-\frac{10}{16}=\frac{6}{16}=\frac{3}{8}_.
\end{align*}
\end{proof}


%%%%%%%%%%%%%%%%%%%%%%%%%%%%%%%%%%%%%%%%%%%%%%%%%%%%%%%%%%%%%%%%%%%%%%%%%%%%%%%%%%%%%%%%%%%%%%%%%%%%%%%%%%%%%%%%%%%%%%%%%%%%%%%%%%%%%

%%%%%%%%%%%%%%%%%%%%%%%%%%%%%%%%%%%%%%%%%%%%%%%%%%%%%%%%%%%%%%%%%%%%%%%%%%%%%%%%%%%%%%%%%%%%%%%%%%%%%%%%%%%%%%%%%%%%%%%%%%%%%%%%%%%%
